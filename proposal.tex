\documentclass{article}


\title{Peer-to-Peer Wikipedia}
\author{Rain Gu and Nick Bradley}

\begin{document}
\maketitle

\section{Introduction and Motivation}
Wikimedia's mission is to bring free educational content to the world.(1) One way
they do this is through Wikipedia an online collaborative encyclopedia hosted by
Wikimedia. Having a single organization manage the content has several implications
for unbiased content. For eaxmple, if there are negative comments about a company
that company could pay to have it removed
which they do this is by supporting the infrastructure required to host online
content such as Wikipedia. This is one of the largest costs and has opportunities
for content to become biased. In the client-server model used by Wikipedia, full
control of the documents lies with the server managers.

We are going to investigate the feasiblity of a peer-to-peer Wikipedia-style document
sharing system.

Wikipedia spends \$\$\$ on hosting costs per year.

\section{Background and Assumptions}
Our primary focus is on making


Peer-to-peer systems are _____. They allow ____
Our network has ____ nodes that have a MTBF of ____ (a node is considered failed if
it does not accept messages from the leader)
The focus is on availability of documents (what is a document?). We will guarantee that
\begin{enumerate}
  \item the content of a new document will be available to any client once it is
  committed by the leader.
  \item the content of a document will not be available to any client if it is
  deleted on any connected node.
\end{enumerate}



\section{Proposed Approach}
We start under the strongest guaratee that any document must be available if any
node is online. This requires every document to be stored on every node. We can
implement our application protocol on top of Raft.

We may then investigate how to relax the availablity guarantee to allow for
scaling to a larger network. May be able to give a probabilistic bound based on uptime:
since uptime is independent, could have that the average uptime of the nodes that
a document is on must be greater than some threshold.

For example, if a document is on 10 nodes with uptimes of about 1 hour, then it
would be equivalent to a document being on 5 nodes with uptimes of 2 hours.

Question: how do you pick whch nodes the document goes on?

Under this approach, need a search method to find the document.


\section{Evaluation Methodology}
Simulate a variety of network setups to evaluate performance

Make sure or guarantees hold on all edge cases
1) will the document propgate if a leader election happens?

\section{Timeline}
\section{References}
1) Wikimedia https://www.wikimedia.org/


\end{document}
