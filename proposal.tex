\documentclass{article}

\usepackage[margin=1in]{geometry}
\usepackage{enumitem}


\title{Peer-to-Peer Wikipedia}
\author{Rain Gu and Nick Bradley}

\begin{document}
\maketitle

\section{Introduction}
Wikipedia is a free online collaborative encyclopedia hosted by Wikimedia and
funded by the non-profit Wikimedia foundation. We propose a peer-to-peer version
of Wikipedia for two primary reasons. First, decentralization helps eliminate
proprietary interests in the system’s infrastructure; instead of trust being
placed in dedicated servers, trust is diffused over all participants. Second,
the need for administration is diminished, since there is no dedicated
infrastructure to manage.

Our project addresses two major issues that arise in a peer-to-peer encyclopedia:
article replication and article discovery. Replication allows articles to be
available despite node failures but requires consistent versioning across replicas.
Discovery of articles is done via a search of the network. The simplest approach
is to traverse the entire network until the required article is found but this
results in a search time that is proportional to the number of nodes and generates
a large quantity of traffic. An alternative approach is to maintain a structure
on top of the physical network to bound the search time. These issues are discussed
in the next section.

\section{Background}
A peer-to-peer network consists of distributed nodes that make some of their
resources available to others without the use of central coordination. Nodes are
both consumers and suppliers of resources and may leave/fail or join the network
at any time. There are three main peer-to-peer architectures: unstructured,
structured and hybrid. Unstructured networks …

Replication of content across nodes in any type of distributed system is important
for ensuring availability. However, replication increases complexity by having
to ensure that content is successfully distributed across a sufficient number of
nodes and that the content has the same version on all nodes. That is, the same
content to be shown to the requestee regardless of the node hosting the article.

…replication on unstructured network => Nodes with sufficient space respond with uptime; master chooses subset; synchronizes versions.

To ensure that versions are consistent across nodes we will employ Interval Tree
Clocks (ITC) \cite{abf08}. We choose ITCs over simpler version vectors \cite{ppr83}
or version stamps \cite{abf02} because it does not require global ids nor global
coordination to manage versions and unlike version stamps, it is suitable for
practical use \cite{abf07}. We will use the core operations: \texttt{fork},
\texttt{event} and \texttt{join} that are used to model causal tracking mechanisms
in ITC. These operators act on stamps (logical clocks) whose structure is a pair
$(i,e)$ where $i$ is an id and $e$ is a version vector. Thus, for our application,
causality is the pointwise partial order: $e \leq e' \textrm{ iff } \forall{k}, e[k] \leq e'[k]$.
We will use the following definitions to implement the operators:

\begin{itemize}
  \item[\textbf{\texttt{fork}}] Clone the causal past of a stamp, outputting a pair of stamps that
  have identical copies of the event component and distinct ids:
  $\texttt{fork}(i,e) = \left((i_{1},e),(i_{2},e)\right)$ such that $i_{2} \ne i_{1}$.

  \item[\textbf{\texttt{event}}] Increments a counter associated to the identity in the stamp:
  $\forall{k} \ne i, e'[k] = e[k] \textrm{ and } e'[i] = e[i] + 1$.

  \item[\textbf{\texttt{join}}] Merge two stamps, producing a new one:
  $\texttt{join}\left((i_{1},e_{1}),(i_{2},e_{2})\right) = (i_{3},e_{3})$ where $e_{3}$
  is the pointwise maximim of $e_{1}$ and $e_{2}$.
\end{itemize}

\noindent
To be a useful service, our peer-to-peer encyclopedia will fulfill two requirements
\begin{enumerate}[
  labelindent=\parindent,
  style=multiline,
  leftmargin=*,
  label=(SG\arabic*)
]
  \item The latest version of every article will be available with high probability
  \item Versioning of articles is consistent across all nodes storing the article
\end{enumerate}

\noindent
We explain how we plan to achieve these requirements in the next section.

\section{Proposed Approach}
\subsection{Article Discovery}

\subsection{Article Replication}
An article must be replicated to a sufficient number of nodes to guarantee (up to
some probability) that the content will be available any time it is modified.

\textbf{Choosing replica nodes.}

\textbf{Determining version number.} If a client wishes to modify an article,
then the client must first find the latest version of the article using the method
described in the article discovery section. If no article is found, than the client
is creating a new article (this is true with high probability because of SG1).
Otherwise, the client modifies the content of the latest version.

We use ITC \footnote{We will use the Go ITC library \cite{itclib}} to guarantee
consistent versioning across replicas. For forking operations we will use the MAC
address of the node to generate the new id.  \\

\noindent
Creating an article
\begin{itemize}[label={}]
  \item Assign a seed version stamp to the newly created article
  \item Fork the version to the chosen replica nodes
\end{itemize}

\noindent
Updating an article
\begin{itemize}[label={}]
  \item Fork the latest version from some node
  \item Modify the content of the article
  \item Sync \footnote{A \texttt{sync} is the atomic composition of \texttt{join}
  followed by \texttt{fork}.} the new version to the chosen replica nodes
\end{itemize}

\noindent
Deleting an article (not implemented: can't guarantee that article is removed
from all nodes)

\section{Evaluation Methodology}


\section{Timeline}


\bibliographystyle{abbrv}
\bibliography{proposal}
\end{document}
