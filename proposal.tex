\documentclass{article}

\usepackage[margin=1in]{geometry}
\usepackage{csquotes}
\usepackage{enumitem}



\title{Peer-to-Peer Wikipedia}
\author{Rain Gu and Nick Bradley}
\date{October 18, 2016}

\begin{document}
\maketitle

\section{Introduction}
Wikipedia is a free online collaborative encyclopedia hosted by Wikimedia and
funded by the non-profit Wikimedia foundation. We propose a peer-to-peer version
of Wikipedia for two primary reasons. First, decentralization helps eliminate
proprietary interests in the system's infrastructure: instead of trust being
placed in dedicated servers, trust is diffused over all participants. Second,
the need for administration is diminished, since there is no dedicated
infrastructure to manage.

Our project addresses two major issues that arise in a peer-to-peer encyclopedia:
article replication and article discovery. Replication allows articles to be
available despite node failures but requires consistent versioning across replicas.
Discovery of articles is done via a search of the network. The simplest approach
is to traverse the entire network until the required article is found, but this
results in a search time that is proportional to the number of nodes and generates
a large quantity of traffic. An alternative approach is to maintain a structure (overlay)
on top of the physical network to bound the search time. These issues are discussed
in the next section.

\section{Background}
A peer-to-peer network consists of distributed nodes that make some of their
resources available to others without the use of central coordination. Nodes are
both consumers and suppliers of resources and may leave/fail or join the network
at any time. Nodes can be organized using overlays. An unstructured overlay is
\enquote{an overlay in which a node relies on its adjacent nodes for delivery of
messages to other nodes in the overlay} \cite{byl08}. A structured overlay is
\enquote{an overlay in which nodes cooperatively maintain routing information
about how to reach all nodes in the overlay} \cite{byl08}. There is a trade-off
between search efficiency and routing maintenance between the two overlays.

Replication of content across nodes in any type of distributed system is important
for ensuring availability. However, replication increases complexity by requiring
that content be successfully distributed across a sufficient number of nodes while also
maintaining consistent versions. That is, the same content must be shown to the
requestee regardless of the node hosting the article.

To ensure that versions are consistent across nodes we will employ Interval Tree
Clocks (ITC) \cite{abf08}. We choose ITCs over simpler version vectors \cite{ppr83}
or version stamps \cite{abf02} because it does not require global IDs or global
coordination to manage versions and, unlike version stamps, it is suitable for
practical use \cite{abf07}. We will use the core operations -- \texttt{fork},
\texttt{event}, and \texttt{join} -- that are used to model causal tracking mechanisms
in ITC. These operators act on stamps (logical clocks) whose structure is a pair
$(i,e)$ where $i$ is an ID and $e$ is a version vector. Thus, for our application,
causality is the pointwise partial order: $e \leq e' \textrm{ iff } \forall{k}, e[k] \leq e'[k]$.
We will use the following definitions to implement the operators:

\begin{itemize}
  \item[\textbf{\texttt{fork}}] Clone the causal past of a stamp, outputting a pair of stamps that
  have identical copies of the event component and distinct IDs:
  $\texttt{fork}(i,e) = \left((i_{1},e),(i_{2},e)\right)$ such that $i_{2} \ne i_{1}$.

  \item[\textbf{\texttt{event}}] Increments a counter associated to the identity in the stamp:
  $\forall{k} \ne i, e'[k] = e[k] \textrm{ and } e'[i] = e[i] + 1$.

  \item[\textbf{\texttt{join}}] Merge two stamps, producing a new one:
  $\texttt{join}\left((i_{1},e_{1}),(i_{2},e_{2})\right) = (i_{3},e_{3})$ where $e_{3}$
  is the pointwise maximim of $e_{1}$ and $e_{2}$.
\end{itemize}

\noindent
To be a useful service, our peer-to-peer encyclopedia will fulfill two requirements
\begin{enumerate}[
  labelindent=\parindent,
  style=multiline,
  leftmargin=*,
  label=(SG\arabic*)
]
  \item The latest version of every article will be available with high probability, and
  \item Versioning of articles is consistent across all nodes storing the article.
\end{enumerate}

\noindent
We explain how we plan to achieve these requirements in the next section.

\section{Proposed Approach}
\subsection{Assumptions}
We make the following assumptions about the system:
\begin{itemize}
  \item Nodes fail (app isn’t running) independently and with the same probability $p$.
  \item There are no Byzantine faults or application bugs.
  \item Messages will not be dropped or corrupted.
  \item Messages may be delayed but will be ordered correctly.
\end{itemize}

\subsection{Article Discovery}
For our initial pass, we will use an unstructured overlay and use a broadcast or
flooding search where each node is queried until the requested article is found. There are
two nice features of this approach: (1) it is simple to implement; and (2) it is
very robust to both churn and query types. However, this method can be very slow
for large networks. One way to improve performace is to use an intelligent search
algorithm which uses the past behaviour of the network by having each peer
autonomously decide which of its peers are likely to answer the query \cite{kgz02}.
We may also evaluate structured overlays where nodes are added and removed such
that each node maintains some information about how to reach other nodes in the
network.

\subsection{Article Replication}
An article must be replicated to a sufficient number of nodes to guarantee (up to
some probability) that the content will be available any time it is modified.

\textbf{Choosing replica nodes.} To guarantee an article is available with probability
$\alpha$, we send the article to $n \geq \lceil \frac{\log{(1-\alpha)}}{\log{p}} \rceil$
 nodes. We keep sending the article to different nodes until we receive $n$ responses
 (there is no harm if the article is on more than $n$ nodes).

\textbf{Determining version number.} If a client wishes to modify an article,
then the client must first find the latest version of the article using the method
described in the article discovery section. If no article is found, then the client
is creating a new article (this is true with high probability because of SG1).
Otherwise, the client modifies the content of the latest version.

We use ITC \footnote{We will use the Go ITC library https://github.com/fgrid/itc} to guarantee
consistent versioning across replicas. For forking operations we will use the MAC
address of the node to generate the new id. \\

\noindent
Creating an article
\begin{itemize}[label={}]
  \item Assign a seed version stamp to the newly created article
  \item Fork the version to the chosen replica nodes
\end{itemize}

\noindent
Modifying an article
\begin{itemize}[label={}]
  \item Fork the latest version from some node
  \item Modify the content of the article
  \item Sync \footnote{A \texttt{sync} is the atomic composition of \texttt{join}
  followed by \texttt{fork}.} the new version to the chosen replica nodes
\end{itemize}

\noindent
Deleting an article (not implemented: can't guarantee that article is removed
from all nodes)

\section{Evaluation Methodology}
To test the algorithms and protocols we designed and implemented, we will build
an application in Go with simple user interface (i.e. HTML). Since p2p Wikipedia
is targeting to be used over internet, we will build several nodes using proxy IP
to simulate the real world scenarios. The top one attribute of the application to
test is SG2: Versioning of articles is consistent across all nodes storing the
articles, since SG1 is achieved mostly by assumption. \\

\noindent
The cases to test are:
\begin{enumerate}
  \item a client read an article.
  \item a client update an article while no other clients updating the same articles in a close time interval.
  \item multiple clients are updating the same article in a close time interval.
  \item a new client(node) joins the system and bootstrap.
\end{enumerate}
For each cases, we will also test some cases with the faulting nodes or network failure.
The other technique we are using would be the GoVector Library to trace the message orders.

\section{Timeline}

\begin{tabular*}{\textwidth}{l @{\extracolsep{\fill}} ll}
  Set up Node Template and Data Structures (i.e. ITC). && Oct 27 \\
  Implement Search \& Read operations and protocols. && Nov 11 \\
  Implement Update operation and protocols. && Nov 20 \\
  Design a simple UI && Nov 23 \\
  Final Encapsulation \& Report && Nov 30 \\
\end{tabular*}


\bibliographystyle{abbrv}
\bibliography{proposal}
\end{document}
